% 下列中文名詞的定義,如果以註解方式關閉取消,
% 則會以系統原先的預設值 (英文) 替代
% 名詞 \prechaptername 預設值為 Chapter
% 名詞 \postchaptername 預設值為空字串
% 名詞 \tablename 預設值為 Table
% 名詞 \figurename 預設值為 Figure


%\renewcommand\prechaptername{第} % 出現在每一章的開頭的「第 x 章」
%\renewcommand\postchaptername{章}

\ctexset{
    chapter = {
        name = {第,章},
        number = \chinese{chapter},
        format = \centering\CJKfamily{kai}\fontsize{20}{20}\selectfont\textbf,  % 設定字體為標楷體並且大小為20pt
        beforeskip = -24pt,  % 章標題前的間距
        afterskip = 6pt,  % 章標題後的間距
    },
    section = {
        name = {第,節},
        number = \chinese{section},
        format = \centering\CJKfamily{kai}\fontsize{18}{18}\selectfont\textbf,  % 調整字體與大小
        beforeskip = 0pt,  % 調整節標題前的空間
        afterskip = 6pt,  % 節標題與正文之間無空行
    },
}

\renewcommand{\tablename}{表} % 在文章中 table caption 會以「表 x」表示
\renewcommand{\figurename}{圖} % 在文章中 figure caption 會以「圖 x」表示

% 下列中文名詞的定義,用於論文固定的各部分之命名 (出現於目錄與該頁標題)
\newcommand{\nameInnerCover}{教授推薦書}
\newcommand{\nameCommitteeForm}{論文口試委員審定書}
\newcommand{\nameCopyrightForm}{授權書}
\newcommand{\nameCabstract}{摘 要}
\newcommand{\nameEabstract}{ABSTRACT}
\newcommand{\nameAckn}{誌 謝}
\newcommand{\nameToc}{目 錄}
\newcommand{\nameTof}{圖 目 錄}
\newcommand{\nameLot}{表 目 錄}
\newcommand{\nameToa}{演算法目錄}
\newcommand{\nameSlist}{符號說明}
\newcommand{\nameRef}{參考文獻}
\newcommand{\nameVita}{自傳}
