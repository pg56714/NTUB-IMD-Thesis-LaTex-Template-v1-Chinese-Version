%
% this file is encoded in utf-8
% v1

\documentclass[landscape,12pt]{article}

% 如果使用 latex + dvipdfmx 工作流程,要特別使用 -l 來指定 landscape 排版
%   dvipdfmx -l yzu_coverbridge.dvi
%
% pdflatex 工作流程則自動選定 landscape 排版
%
% 需要 LaTeX CJK 套件 4.6.0 或更新的版本
% 
\usepackage{CJK}

\usepackage{geometry}
\geometry{verbose,a4paper,landscape, tmargin=2.5cm, bmargin=3cm, lmargin=2.5cm, rmargin=3cm}


\ifx\pdfoutput\undefined
  % not running pdftex
  \usepackage[dvipdfm]{graphicx}
\else
  \ifx\pdfoutput\relax
    % not running pdftex
    \usepackage[dvipdfm]{graphicx}
  \else
    % running pdftex, with...
    \ifnum\pdfoutput>0
      % ... PDF output
      \usepackage[pdftex]{graphicx}
    \else
      %...DVI output
      \usepackage[dvipdfm]{graphicx}
    \fi
  \fi
\fi


\usepackage{rotating} %旋轉「畢業年份」阿拉伯數字所需之套件
\usepackage{CJKvert} %旋轉中文字所需之套件

\parindent = 0Pt  % 段首內縮由 CJK 控制,所以這裡就設成不內縮

\begin{document}
\thispagestyle{empty} % no page number
%
\begin{CJK}{UTF8}{bkai} 
%所選之中文字體須有配合直排之 .fdx 檔,否則所穿插的英、數符號與中文字會對不齊
%
% varible place holders
% 此處只是預設值,不需更改此處
% 請更改 my_names.tex 內容

\newcommand\cTitle{}
\newcommand\eTitle{}
\newcommand\myCname{}
\newcommand\advisorCnameA{}
\newcommand\univCname{}
\newcommand\deptCname{}
\newcommand\degreeCname{}
\newcommand\cYear{}
\newcommand\cMonth{}

%
% 中文直排 (反時鐘旋轉每一中文字 90 度)
\CJKvert
%
% font size:
% \large (14pt) < \Large (18pt) < \LARGE (20pt) < \huge (24pt)< \Huge (24 pt)
%
% 重新定義中文字距,避免旋轉後之中文字過密
\renewcommand{\CJKglue}{\hskip 0.5pt plus 0.08\baselineskip}
%
% 取得個人資料
%
% this file is encoded in utf-8
% v1
% 填入你的論文題目、姓名等資料
% 如果題目內有必須以數學模式表示的符號,請用 \mbox{} 包住數學模式

% 論文題目 (中文)
\renewcommand\cTitle{%我的碩士論文題目 
論文題目
}

% 論文題目 (英文)
\renewcommand\eTitle{%My Thesis Title  
My Thesis Title
% My Thesis Title  \mbox{$\cal{H}_\infty$} and \mbox{Al$_x$Ga$_{1-x}$As}
}

% 我的姓名 (中文)
\renewcommand\myCname{你(妳)的名字}

% 指導教授A的姓名 (中文)
\renewcommand\advisorCnameA{指導教授的姓名\ 博士}

% 校名 (中文)
\renewcommand\univCname{國立臺北商業大學管理學院}

% 系所名 (中文)
\renewcommand\deptCname{資訊管理系人工智慧與商業應用碩士班}

% 學位名 (中文)
\renewcommand\degreeCname{碩士學位}

% 口試年份 (中文、民國)
\renewcommand\cYear{一一四}

% 口試月份 (中文)
\renewcommand\cMonth{六} 

%畢業級別;用於書背列印;若無此需要可忽略
\newcommand\GraduationClass{114}

%%%%%%%%%%%%%%%%%%%%%%
%
%
%
% 開始排版
%
%
% 口試年份 (畢業級別)
\makebox[1cm][l]{%
\begin{sideways}\GraduationClass\end{sideways}%
}%
%
% 顯示論文種類
\makebox[2.5cm][c]{\Large{\degreeCname 論文}}%
%
\hfill
% 論文篇名 (如果篇名太長,會自動折行)
% 因為旋轉後的中文,在 makebox 與 parbox 裏的基線有些微差距 (原因不明),使得以 parbox 編排的論文篇名與校所名會比其他以 makebox 編排的部分低 (直立後偏左)。所以用 raisebox 微調
\newcommand\myFineTune{1.75pt} % 微調偏移量
%
\raisebox{\myFineTune}{\parbox{13.5cm}{\renewcommand\baselinestretch{0.5}\Large{\cTitle}}%
}
\hfill
%
% 校、所名
\raisebox{\myFineTune}{%
\parbox{3.1cm}{% 校名與系所名要很靠近,所以用 \tiny 小號字體取得小行距
\tiny \makebox[3cm][s]{\small{\univCname}}\\%
\makebox[3cm][s]{\footnotesize{\deptCname}}%
}%
}%
%
%
\hspace{0.8cm}
%
\makebox[2cm][s]{\Large{\myCname}}%
%
\end{CJK}
\end{document} 
