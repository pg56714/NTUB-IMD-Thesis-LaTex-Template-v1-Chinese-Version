%
% this file is encoded in utf-8
% v1

\pagestyle{plain}  % 前幾頁不顯示 fancyhdr

% 無須修改本檔內容,除非校方修改了
% 封面、書名頁、中文摘要、英文摘要、誌謝、目錄、表目錄、圖目錄、符號說明
% 等頁之格式

% make the line spacing in effect
% \renewcommand{\baselinestretch}{\mybaselinestretch}
% \large % it needs a font size changing command to be effective

% default variables definitions
% 注意!!此處只是預設值,不需更改此處
% 請更改 my_names.tex 內容
\newcommand\cTitle{論文題目}
\newcommand\eTitle{MY THESIS TITLE}
\newcommand\myCname{OOO}
\newcommand\advisorCnameA{OOO\ 博士}
\newcommand\univCname{國立臺灣大學}
\newcommand\deptCname{資訊工程研究所}
\newcommand\degreeCname{碩士}
\newcommand\cYear{一一四}
\newcommand\cMonth{六}

 % user's names; to replace those default variable definitions
%
% this file is encoded in utf-8
% v1
% 填入你的論文題目、姓名等資料
% 如果題目內有必須以數學模式表示的符號,請用 \mbox{} 包住數學模式

% 論文題目 (中文)
\renewcommand\cTitle{%我的碩士論文題目 
論文題目
}

% 論文題目 (英文)
\renewcommand\eTitle{%My Thesis Title  
My Thesis Title
% My Thesis Title  \mbox{$\cal{H}_\infty$} and \mbox{Al$_x$Ga$_{1-x}$As}
}

% 我的姓名 (中文)
\renewcommand\myCname{你(妳)的名字}

% 指導教授A的姓名 (中文)
\renewcommand\advisorCnameA{指導教授的姓名\ 博士}

% 校名 (中文)
\renewcommand\univCname{國立臺北商業大學管理學院}

% 系所名 (中文)
\renewcommand\deptCname{資訊管理系人工智慧與商業應用碩士班}

% 學位名 (中文)
\renewcommand\degreeCname{碩士學位}

% 口試年份 (中文、民國)
\renewcommand\cYear{一一四}

% 口試月份 (中文)
\renewcommand\cMonth{六} 

%畢業級別;用於書背列印;若無此需要可忽略
\newcommand\GraduationClass{114}

%%%%%%%%%%%%%%%%%%%%%%
%%%%%%%%%%%%%%%%%%%%%%%%%%%%%%%
%       ntust cover 封面
%%%%%%%%%%%%%%%%%%%%%%%%%%%%%%%
%
% this file is encoded in utf-8
% v1

\newgeometry{top=2cm, bottom=2cm, left=2cm, right=2cm}

%%%%%%%%%%%%%%%%%%%%%%%%%%%%%%%
%       ntust cover 封面
%%%%%%%%%%%%%%%%%%%%%%%%%%%%%%%
%
\begin{titlepage}
% no page number
% next page will be page 1

% aligned to the center of the page horizontally
\begin{center}
% font size (relative to 12 pt):
% \large (14pt) < \Large (18pt) < \LARGE (20pt) < \huge (24pt)< \Huge (24 pt)
%
% 校名與系所名
% \vspace*{0cm}
{\CJKfamily{kai}\fontsize{26pt}{26pt}\selectfont\textbf{\univCname}}\\ % 校名,26pt
\vspace{0.25cm}
{\CJKfamily{kai}\fontsize{24pt}{36pt}\selectfont\textbf{\deptCname}}\\ % 系所名,24pt
\vspace{0.25cm}
{\CJKfamily{kai}\fontsize{24pt}{36pt}\selectfont\textbf{\degreeCname 論文}}\\ % 論文種類,24pt
\vspace{0.25cm}
%
\vspace{18pt}
\vspace{18pt}
\vspace{18pt}
\vspace{18pt}
\vspace{18pt}
%
{\CJKfamily{kai}\fontsize{24pt}{36pt}\selectfont\textbf{\cTitle}}\\ % 論文種類,24pt
\fontsize{22pt}{22pt}\selectfont{\eTitle}\\ % 英文題目,20pt 或 22pt,Times New Roman
%
\vspace{12pt}
\vspace{12pt}
\vspace{12pt}
\vspace{12pt}
\vspace{12pt}
\vspace{12pt}
\vspace{12pt}
\vspace{12pt}
\vspace{12pt}
\vspace{12pt}
%
% 研究生與指導教授信息
{\CJKfamily{kai}\fontsize{18pt}{18pt}\selectfont\textbf{{研究生:\Large{\myCname}}}}\\ % 研究生,18pt
\vspace{18pt}
\vspace{18pt}
\vspace{18pt}
\vspace{18pt}
{\CJKfamily{kai}\fontsize{18pt}{18pt}\selectfont\textbf{指導教授:\Large{\advisorCnameA}}}\\ % 指導教授A,18pt
\vspace{18pt}
\vspace{18pt}
\vspace{18pt}
% 顯示日期 18pt
{\CJKfamily{kai}\fontsize{18pt}{18pt}\selectfont\textbf{中華民國\cYear 年\cMonth 月}}\\
%
\end{center}
% 恢復原設置
% \renewcommand{\baselinestretch}{\mybaselinestretch}   %恢復原設定
% % restore the font size to normal
% \normalsize
\end{titlepage}
%%%%%%%%%%%%%%

%%%%%%%%%%%%%%

\newgeometry{top=3cm, bottom=3cm, left=3.5cm, right=3cm}

%% 從摘要到本文之前的部份以小寫羅馬數字印頁碼
% 但是從「書名頁」(但不印頁碼) 就開始計算
%\setcounter{page}{1}
\pagenumbering{Roman}
%\pagenumbering{arabic}

% 判斷是否要浮水印?
\ifx\mywatermark\undefined 
  \thispagestyle{empty}  % 無頁碼、無 header (無浮水印)
\else
  \thispagestyle{EmptyWaterMarkPage} % 無頁碼、有浮水印
\fi

%%%%%%%%%%%%%%%%%%%%%%%%%%%%%%%%%%%%%%%%%%%%%%%%%%%%%%%%%%%%%%%


%%%%%%%%%%%%%%%%%%%%%%%%%%%%%%%%%%%%%%%%%%%%%%%%%%%%%%%%%%%%%%%%%%%%%
%%%%%%%%%%%%%%%%%%%%%%%%%%%%%%%
%       論文口試委員審定書、無違反學術倫理聲明書(不顯示頁碼、目錄)
%%%%%%%%%%%%%%%%%%%%%%%%%%%%%%%
%
% insert the printed standard form when the thesis is ready to bind
% 在口試完成後,再將已簽名的審定書、聲明書放入以便裝訂
% create an entry in table of contents for 審定書、聲明書

% % 插入審定書 PDF(假設檔名為 approval.pdf)
% \cleardoublepage
% \includepdf[pages=-, pagecommand={\thispagestyle{empty}}]{frontpages/forms/口試審定書-簽名版.pdf}

% % 插入學術倫理聲明書 PDF(假設檔名為 ethics.pdf)
% \cleardoublepage
% \includepdf[pages=-, pagecommand={\thispagestyle{empty}}]{frontpages/forms/無違反學術倫理聲明書-簽名版.pdf}

% ↓ 在插入完兩個 PDF 後,重設頁碼計數
% \cleardoublepage
% \setcounter{page}{1}
% \pagenumbering{Roman}

%%%%%%%%%%%%%%%%%%%%%%%%%%%%%%%
%       中文摘要 
%%%%%%%%%%%%%%%%%%%%%%%%%%%%%%%
%
% aligned to the center of the page
\chapter*{\mdseries\nameCabstract}
% create an entry in table of contents for 中文摘要
% \addcontentsline{toc}{chapter}{ \nameCabstract}
\addcontentsline{toc}{chapter}{\nameCabstract}
\thispagestyle{plain}  % 無 header,但在浮水印模式下會有浮水印
% \vspace*{0.5cm}
% Resume the line spacing to the desired setting
\renewcommand{\baselinestretch}{\mybaselinestretch}   %恢復原設定
%it needs a font size changing command to be effective
% restore the font size to normal
\normalsize
%%%%%%%%%%%%%

分散式詢問及監督系統主要被用於分散式資料如檔案或是紀錄的維護上。網路內的使用者可以自系統中查詢所需資料如總和或平均值,若同時將所有原始資料做傳遞及運算,將耗費相當大的網路頻寬及運算資源。於是,內網路聚集技術被提出來降低分散式詢問及監督系統的負擔。然而,這個技術卻容易遭受安全威脅。過去的研究大多假設資料來源為可信任,並針對聚集架構進行安全性的研究。然而,我們認為聚集查詢結果應該在面對惡意攻擊者將錯誤資料置入資料串流進行聚集前,就應該具備強健的容錯能力。傳統上,一個強健的估計值被定義為即使資料來源有誤時,亦能維持一定程度正確性的聚集結果。許多常見的強健估計值是建立在有序統計學上,因此,我們將重心放在內網路計算上之有序統計的可驗證技術。此技術的挑戰為在網路遭受惡意團體介入聚集程序時,仍能確保聚集結果或近似結果的準確性。
	
\vspace{1em}
\vspace{1em}
\vspace{1em}
\noindent 關鍵字:OOO

%%%%%%%%%%%%%%%%%%%%%%%%%%%%%%%
%       英文摘要 
%%%%%%%%%%%%%%%%%%%%%%%%%%%%%%%
%
% aligned to the center of the page
\chapter*{\mdseries\nameEabstract}
% create an entry in table of contents for 中文摘要
% \addcontentsline{toc}{chapter}{ \nameEabstract}
\addcontentsline{toc}{chapter}{\nameEabstract}
\thispagestyle{plain}  % 無 header,但在浮水印模式下會有浮水印
% \vspace*{0.5cm}
% Resume the line spacing to the desired setting
\renewcommand{\baselinestretch}{1.5}   %恢復原設定
%it needs a font size changing command to be effective
% restore the font size to normal
\normalsize
%%%%%%%%%%%%%
Distributed querying and monitoring systems have been widely studied in recent years. These systems aim to maintain data sources, such as data set or log files, and allow users to query over those data sources. When the data sources are highly related and users only care some statistic results, like the sum or the average, it is consumed to transmit all data sources via the network. To minimize the network consumption, in-network aggregation technique is proposed. However, this technique is subject to some known attacks, such as the injection attack and the pollution attack. Prior works only considered the settings that data sources are trusted while the network is not. We study the way to relax the limitation and guarantee the aggregate queries robust to malicious or faulty data sources (also called polluted data sources). 

\vspace{1em}
\vspace{1em}
\vspace{1em}
\noindent Keywords:OOO

%%%%%%%%%%%%%%%%%%%%%%%%%%%%%%%
%       誌謝 
%%%%%%%%%%%%%%%%%%%%%%%%%%%%%%%
%
% Acknowledgment
% aligned to the center of the page
\chapter*{\mdseries\nameAckn}
% create an entry in table of contents for 中文摘要
% \addcontentsline{toc}{chapter}{ \nameAckn}
\addcontentsline{toc}{chapter}{\nameAckn}
\thispagestyle{plain}  % 無 header,但在浮水印模式下會有浮水印
% \vspace*{0.5cm}
% Resume the line spacing to the desired setting
\renewcommand{\baselinestretch}{\mybaselinestretch}   %恢復原設定
%it needs a font size changing command to be effective
% restore the font size to normal
\normalsize
%%%%%%%%%%%%%

	首先誠摯的感謝指導教授陳明明博士,老師悉心的教導使我得以一窺WSN的深奧,不時的討論並指點我正確的方向,使我在這些年中獲益匪淺。老師對學問的嚴謹更是我輩學習的典範。
    本論文的完成另外亦得感謝老師們大力協助。因為有你們的體諒及幫忙,使得本論文能夠更完整而嚴謹。
 兩年裡的日子,實驗室裡共同的生活點滴,學術上的討論、言不及義的閒扯、讓人又愛又怕的宵夜、趕作業的革命情感、因為睡太晚而遮遮掩掩閃進實驗室........,感謝眾位學長姐、同學、學弟妹的共同砥礪,你/妳們的陪伴讓兩年的研究生活變得絢麗多彩。
	最後絕對不能忘記最了解、最支持我的家人─我的父親、母親及姊姊,在我喪失動力之時,隨時都能給予我心靈上無窮盡的關心與鼓勵,讓我有勇氣堅持到最後,完成研究的旅途。還有很多曾經幫助過我的朋友,因為有大家的幫助,我才能有今天的成果。想要感謝的人真的太多太多,就只有感謝上天了!






% Resume the line spacing to the desired setting
\renewcommand\addvspace[1]{}

%%%%%%%%%%%%%%%%%%%%%%%%%%%%%%%
%       目錄 
%%%%%%%%%%%%%%%%%%%%%%%%%%%%%%%
%
% Table of contents
\newpage
\renewcommand{\contentsname}{\mdseries\nameToc}
\phantomsection % for hyperref to register this
% \addcontentsline{toc}{chapter}{ \nameToc}
\addcontentsline{toc}{chapter}{\nameToc}
\tableofcontents



%%%%%%%%%%%%%%%%%%%%%%%%%%%%%%%
%       圖目錄 
%%%%%%%%%%%%%%%%%%%%%%%%%%%%%%%
%
% List of Figures
\newcommand{\loflabel}{圖}
\newpage
\renewcommand{\numberline}[1]{\loflabel~#1\hspace*{1em}}
\renewcommand{\listfigurename}{\mdseries\nameTof}
\phantomsection % for hyperref to register this
% \addcontentsline{toc}{chapter}{ \nameTof}
\addcontentsline{toc}{chapter}{\nameTof}
\listoffigures


%%%%%%%%%%%%%%%%%%%%%%%%%%%%%%%
%       表目錄 
%%%%%%%%%%%%%%%%%%%%%%%%%%%%%%%
%
% List of Tables
\newcommand{\lotlabel}{表}
\newpage
\renewcommand{\numberline}[1]{\lotlabel~#1\hspace*{1em}}
\renewcommand{\listtablename}{\mdseries\nameLot}
\phantomsection % for hyperref to register this
% \addcontentsline{toc}{chapter}{ \nameLot}
\addcontentsline{toc}{chapter}{\nameLot}
\listoftables

%%%%%%%%%%%%%%%%%%%%%%%%%%%%%%%
%       演算法目錄 
%%%%%%%%%%%%%%%%%%%%%%%%%%%%%%%
%
% List of Figures
%\newpage
%\renewcommand{\listalgorithmname}{\protect\makebox[5cm][s]{\nameToa}}
%\makebox{} is fragile; need protect
%\addcontentsline{toc}{chapter}{\nameToa}
%\listofalgorithms


%%%%%%%%%%%%%%%%%%%%%%%%%%%%%%%
%       符號說明 
%%%%%%%%%%%%%%%%%%%%%%%%%%%%%%%
%
% Symbol list
% define new environment, based on standard description environment
% adapted from p.60~64, <<The LaTeX Companion>>, 1994, ISBN 0-201-54199-8
%\newcommand{\SymEntryLabel}[1]%
% {\makebox[3cm][l]{#1}}
%
%\newenvironment{SymEntry}
%   {\begin{list}{}%
%       {\renewcommand{\makelabel}{\SymEntryLabel}%
%        \setlength{\labelwidth}{3cm}%
%        \setlength{\leftmargin}{\labelwidth}%
%        }%
%   }%
%   {\end{list}}
%%
%\newpage
%\chapter*{\protect\makebox[5cm][s]{\nameSlist}} %\makebox{} is fragile; need protect
%\addcontentsline{toc}{chapter}{\nameSlist}
%%
% this file is encoded in utf-8
% v1
%  各符號以 \item[] 包住,然後接著寫說明
% 如果符號是數學符號,應以數學模式表示,以取得正確的字體
% 如果符號本身帶有方括號,則此符號可以用大括號 {} 包住保護
\begin{SymEntry}

\item[OLED]
Organic Light Emitting Diode

\item[$E$]
energy

\item[$e$]
the absolute value of the electron charge, $1.60\times10^{-19}\,\text{C}$
 
\item[$\mathscr{E}$]
electric field strength (V/cm)

\item[{$A[i,j]$}]
the  element of the matrix $A$ at $i$-th row, $j$-th column\\
矩陣 $A$ 的第 $i$ 列,第 $j$ 行的元素

\end{SymEntry}

% Resume the line spacing to the desired setting
\renewcommand{\baselinestretch}{\mybaselinestretch}   %恢復原設定
%it needs a font size changing command to be effective
% restore the font size to normal
\normalsize

%% 論文本體頁碼回復為阿拉伯數字計頁,並從頭起算
\newpage
\setcounter{page}{1}
\pagenumbering{arabic}

\pagestyle{fancy}
\fancyhf{}
\fancyfoot[C]{\ifnum\value{page}>0 第 \thepage 頁,共 \pageref{LastPage} 頁\fi}

% 確保 plain 頁碼格式(如章節首頁)也適用 fancyhdr
\fancypagestyle{plain}{%
    \fancyhf{}
    \fancyfoot[C]{第 \thepage 頁,共 \pageref{LastPage} 頁}
}
%%%%%%%%%%%%%%%%%%%%%%%%%%%%%%%%