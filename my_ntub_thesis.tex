%
% this file is encoded in utf-8
% v1

\documentclass[12pt,a4paper]{ntub_report}

\usepackage{fontspec}   % 加這個就可以設定字體 
\usepackage{xeCJK}      % 讓中英文字體分開設置

% fontset=none因本地端字型有Bug,且也不需用到ctex字體,主要用在修改成章跟節(目錄以及標題)
\usepackage[fontset=none, UTF8, heading=true]{ctex}

% \usepackage{showframe} % 查看排版debug用 makes the page borders visible

%設定主要字型,也就是英文字型
\setmainfont{Times New Roman}[
  BoldFont={Times New Roman Bold},
  ItalicFont={Times New Roman Italic},
  BoldItalicFont={Times New Roman Bold Italic}
]

%設定中文字型
%參考 https://www.overleaf.com/learn/latex/Questions/What_OTF/TTF_fonts_are_supported_via_fontspec%3F#Chinese
\setCJKmainfont{mingliu.ttc}[AutoFakeBold=3, Script=Default]% 設定內文為新細明體(沒有內建的),使用W11新細明體字型
\setCJKmonofont{mingliu.ttc}
\setCJKfamilyfont{kai}{kaiu.ttf}[AutoFakeBold=3]% 設定標楷體直接使用TW-Kai也是一樣,使用W11標楷體字型

% 設置段落自動換行
\XeTeXlinebreaklocale "zh"
\XeTeXlinebreakskip = 0pt plus 1pt

% 目錄跟自動編號的深度(章、節)
\setcounter{tocdepth}{2}
\setcounter{secnumdepth}{2}

% https://www.overleaf.com/learn/latex/Hyperlinks
\usepackage{hyperref}
\hypersetup{
    colorlinks=true,
    linkcolor=black, 
    filecolor=black,      
    urlcolor=black,
    citecolor=black,
    pdfpagemode=FullScreen,
    linktoc=all, % 強制所有標題鏈接有效
    pdftitle={基於深度學習的多模型協同影像處理系統研究}, %之後改成論文題目
}

% % debug
% \hypersetup{
%     colorlinks=true,
%     linkcolor=blue,
%     filecolor=blue,
%     urlcolor=blue,
%     citecolor=blue,
%     pdfpagemode=FullScreen,
%     linktoc=all,
%     pdftitle={thesis}
% }

\input{common_env}  %基本的環境設定  無需改變

%參考文獻,記得在bib加上 keywords  = {chinese} or {english}
\usepackage[style=authoryear,sorting=nyt]{biblatex} % 'backend=biber' is the default、nyt = name, year, title
\addbibresource{my_bib.bib}

% 不顯示 type 字段(如 Doctoral dissertation)
\DeclareFieldFormat{type}{}

% 中文標點
\newcommand{\zhcomma}{,}
\newcommand{\zhperiod}{。}

% @miscc 只處理 keywords=chinese
\DeclareBibliographyDriver{miscc}{%
  \usebibmacro{bibindex}%
  \usebibmacro{begentry}%
  \ifkeyword{chinese}{%
    \printnames{author}\zhcomma
    \printfield{year}\zhcomma
    「\printfield{title}」\zhcomma
    \printfield{note}%
    \zhperiod
  }
}

\DefineBibliographyStrings{english}{
  in = {}
}
\renewcommand*{\nameyeardelim}{\addcomma\space}

\DeclareFieldFormat[article,inproceedings,online]{title}{“#1”}
\DeclareLabeldate{%
  \field{year}
  \field{extradate}
}
\DeclareFieldFormat{labelyear}{#1\printfield{extradate}}

\DeclareBibliographyDriver{article}{%
  \usebibmacro{bibindex}%
  \usebibmacro{begentry}%
  \printnames{author}\adddot\addspace%
  \printfield{labelyear}\adddot\addspace
  \printfield{title}\adddot\addspace%
  \printfield{journaltitle}\addcomma\addspace%
  \printfield{volume}\adddot\addspace%
  \iffieldundef{number}{}{%
    \addcomma\addspace%
    no.~\printfield{number}%
  }%
  \iffieldundef{pages}{}{%
    \addcomma\addspace%
    \printfield{pages}%
  }%
  \finentry%
}

\DeclareBibliographyDriver{inproceedings}{%
  \usebibmacro{bibindex}%
  \usebibmacro{begentry}%
  \printnames{author}\adddot\addspace%
  \printfield{labelyear}\adddot\addspace
  \printfield{title}\addcomma\addspace%
  \printfield{booktitle}%
  \iffieldundef{publisher}{}{%
    \addcomma\addspace%
    \printlist{publisher}%
  }%
  \iffieldundef{pages}{}{%
    \addcomma\addspace%
    \printfield{pages}%
  }%
  \finentry%
}

\DeclareBibliographyDriver{online}{%
  \usebibmacro{bibindex}%
  \usebibmacro{begentry}%
  \printnames{author}\adddot\addspace%
  \printfield{labelyear}\adddot\addspace
  \printfield{title}\adddot\addspace%
  original source: \texttt{\thefield{url}}\adddot\addspace
  \printfield{note}%
  \finentry%
}

\usepackage{fancyhdr} % 引入 fancyhdr 套件
\usepackage{lastpage} % 引入 lastpage 套件來獲取最後一頁的頁碼

\begin{document}


	% 下列中文名詞的定義,如果以註解方式關閉取消,
% 則會以系統原先的預設值 (英文) 替代
% 名詞 \prechaptername 預設值為 Chapter
% 名詞 \postchaptername 預設值為空字串
% 名詞 \tablename 預設值為 Table
% 名詞 \figurename 預設值為 Figure


%\renewcommand\prechaptername{第} % 出現在每一章的開頭的「第 x 章」
%\renewcommand\postchaptername{章}

\ctexset{
    chapter = {
        name = {第,章},
        number = \chinese{chapter},
        format = \centering\CJKfamily{kai}\fontsize{20}{20}\selectfont\textbf,  % 設定字體為標楷體並且大小為20pt
        beforeskip = -24pt,  % 章標題前的間距
        afterskip = 0pt,  % 章標題後的間距
    },
    section = {
        name = {第,節},
        number = \chinese{section},
        format = \centering\CJKfamily{kai}\fontsize{18}{18}\selectfont\textbf,  % 調整字體與大小
        beforeskip = 0pt,  % 調整節標題前的空間
        afterskip = 0pt,  % 節標題與正文之間無空行
    },
    subsection = {
        format = \raggedright\CJKfamily{kai}\fontsize{16}{16}\selectfont\textbf,
        beforeskip = 0pt,
        afterskip = 0pt,
    },
    subsubsection = {
        format = \raggedright\CJKfamily{kai}\fontsize{14}{14}\selectfont\textbf,
        beforeskip = 0pt,
        afterskip = 0pt,
    },
}

\renewcommand{\tablename}{表} % 在文章中 table caption 會以「表 x」表示
\renewcommand{\figurename}{圖} % 在文章中 figure caption 會以「圖 x」表示

% 下列中文名詞的定義,用於論文固定的各部分之命名 (出現於目錄與該頁標題)
\newcommand{\nameInnerCover}{教授推薦書}
\newcommand{\nameCommitteeForm}{論文口試委員審定書}
\newcommand{\nameCopyrightForm}{授權書}
\newcommand{\nameCabstract}{\hspace{1em}摘\hspace{1em}要}
\newcommand{\nameEabstract}{\hspace{1em}ABSTRACT}
\newcommand{\nameAckn}{\hspace{1em}誌\hspace{1em}謝}
\newcommand{\nameToc}{\hspace{1em}目\hspace{1em}錄}
\newcommand{\nameTof}{\hspace{1em}圖\hspace{1em}目\hspace{1em}錄}
\newcommand{\nameLot}{\hspace{1em}表\hspace{1em}目\hspace{1em}錄}
\newcommand{\nameToa}{演算法目錄}
\newcommand{\nameSlist}{符號說明}
\newcommand{\nameRef}{參考文獻}
\newcommand{\nameVita}{自傳}
 %在此檔案處定義文章中的中文名詞

	%---------------------------------------------------------------------------------------------------------
	%%% 以下是載入前頁、本文、後頁

	%---------------------------------------------------------------------------------------------------------
	% front matter 前頁
	% 包括封面、書名頁、中文摘要、英文摘要、誌謝、目錄、表目錄、圖目錄、符號說明
	% 在撰寫各章草稿時,可以把此部份「關掉」,以節省無謂的編譯時間。
	% 實際內容由
	%    my_names.tex, my_cabstract.tex, my_eabstract.tex, my_ackn.tex, my_symbols.tex
	% 決定
	% ntust_frontpages.tex 此檔只提供整體架構的定義,不需更動
	% 在撰寫各章草稿時,可以把此部份「關掉」,以節省無謂的編譯時間。
	
	%
% this file is encoded in utf-8
% v1

% 無須修改本檔內容,除非校方修改了
% 封面、書名頁、中文摘要、英文摘要、誌謝、目錄、表目錄、圖目錄、符號說明
% 等頁之格式

% make the line spacing in effect
% \renewcommand{\baselinestretch}{\mybaselinestretch}
% \large % it needs a font size changing command to be effective

% default variables definitions
% 注意!!此處只是預設值,不需更改此處
% 請更改 my_names.tex 內容
\newcommand\cTitle{論文題目}
\newcommand\eTitle{MY THESIS TITLE}
\newcommand\myCname{OOO}
\newcommand\advisorCnameA{OOO\ 博士}
\newcommand\univCname{國立臺灣大學}
\newcommand\deptCname{資訊工程研究所}
\newcommand\degreeCname{碩士}
\newcommand\cYear{一一四}
\newcommand\cMonth{六}

 % user's names; to replace those default variable definitions
%
% this file is encoded in utf-8
% v1
% 填入你的論文題目、姓名等資料
% 如果題目內有必須以數學模式表示的符號,請用 \mbox{} 包住數學模式

% 論文題目 (中文)
\renewcommand\cTitle{%我的碩士論文題目 
論文題目
}

% 論文題目 (英文)
\renewcommand\eTitle{%My Thesis Title  
My Thesis Title
% My Thesis Title  \mbox{$\cal{H}_\infty$} and \mbox{Al$_x$Ga$_{1-x}$As}
}

% 我的姓名 (中文)
\renewcommand\myCname{你(妳)的名字}

% 指導教授A的姓名 (中文)
\renewcommand\advisorCnameA{指導教授的姓名\ 博士}

% 校名 (中文)
\renewcommand\univCname{國立臺北商業大學管理學院}

% 系所名 (中文)
\renewcommand\deptCname{資訊管理系人工智慧與商業應用碩士班}

% 學位名 (中文)
\renewcommand\degreeCname{碩士學位}

% 口試年份 (中文、民國)
\renewcommand\cYear{一一四}

% 口試月份 (中文)
\renewcommand\cMonth{六} 

%畢業級別;用於書背列印;若無此需要可忽略
\newcommand\GraduationClass{114}

%%%%%%%%%%%%%%%%%%%%%%
%%%%%%%%%%%%%%%%%%%%%%%%%%%%%%%
%       ntust cover 封面
%%%%%%%%%%%%%%%%%%%%%%%%%%%%%%%
%
% this file is encoded in utf-8
% v1

\newgeometry{top=2cm, bottom=2cm, left=2cm, right=2cm}

%%%%%%%%%%%%%%%%%%%%%%%%%%%%%%%
%       ntust cover 封面
%%%%%%%%%%%%%%%%%%%%%%%%%%%%%%%
%
\begin{titlepage}
% no page number
% next page will be page 1

% aligned to the center of the page horizontally
\begin{center}
% font size (relative to 12 pt):
% \large (14pt) < \Large (18pt) < \LARGE (20pt) < \huge (24pt)< \Huge (24 pt)
%
% 校名與系所名
% \vspace*{0cm}
{\CJKfamily{kai}\fontsize{26pt}{26pt}\selectfont\textbf{\univCname}}\\ % 校名,26pt
\vspace{0.25cm}
{\CJKfamily{kai}\fontsize{24pt}{36pt}\selectfont\textbf{\deptCname}}\\ % 系所名,24pt
\vspace{0.25cm}
{\CJKfamily{kai}\fontsize{24pt}{36pt}\selectfont\textbf{\degreeCname 論文}}\\ % 論文種類,24pt
\vspace{0.25cm}
%
\vspace{18pt}
\vspace{18pt}
\vspace{18pt}
\vspace{18pt}
\vspace{18pt}
%
{\CJKfamily{kai}\fontsize{24pt}{36pt}\selectfont\textbf{\cTitle}}\\ % 論文種類,24pt
\fontsize{22pt}{22pt}\selectfont{\eTitle}\\ % 英文題目,20pt 或 22pt,Times New Roman
%
\vspace{12pt}
\vspace{12pt}
\vspace{12pt}
\vspace{12pt}
\vspace{12pt}
\vspace{12pt}
\vspace{12pt}
\vspace{12pt}
\vspace{12pt}
\vspace{12pt}
%
% 研究生與指導教授信息
{\CJKfamily{kai}\fontsize{18pt}{18pt}\selectfont\textbf{{研究生:\Large{\myCname}}}}\\ % 研究生,18pt
\vspace{18pt}
\vspace{18pt}
\vspace{18pt}
\vspace{18pt}
{\CJKfamily{kai}\fontsize{18pt}{18pt}\selectfont\textbf{指導教授:\Large{\advisorCnameA}}}\\ % 指導教授A,18pt
\vspace{18pt}
\vspace{18pt}
\vspace{18pt}
% 顯示日期 18pt
{\CJKfamily{kai}\fontsize{18pt}{18pt}\selectfont\textbf{中華民國\cYear 年\cMonth 月}}\\
%
\end{center}
% 恢復原設置
% \renewcommand{\baselinestretch}{\mybaselinestretch}   %恢復原設定
% % restore the font size to normal
% \normalsize
\end{titlepage}
%%%%%%%%%%%%%%

%%%%%%%%%%%%%%

\newgeometry{top=3cm, bottom=3cm, left=3.5cm, right=3cm}

%% 從摘要到本文之前的部份以小寫羅馬數字印頁碼
% 但是從「書名頁」(但不印頁碼) 就開始計算
%\setcounter{page}{1}
\pagenumbering{Roman}
%\pagenumbering{arabic}
%%%%%%%%%%%%%%%%%%%%%%%%%%%%%%%
%       指導教授推薦書 
%%%%%%%%%%%%%%%%%%%%%%%%%%%%%%%
%
% insert the printed standard form when the thesis is ready to bind
% 在口試完成後,再將已簽名的推薦書放入以便裝訂
% create an entry in table of contents for 推薦書
% 目前送出空白頁
%\newpage{\thispagestyle{empty}\addcontentsline{toc}{chapter}{\nameInnerCover}\mbox{}\clearpage}%
%\newpage

% 判斷是否要浮水印?
\ifx\mywatermark\undefined 
  \thispagestyle{empty}  % 無頁碼、無 header (無浮水印)
\else
  \thispagestyle{EmptyWaterMarkPage} % 無頁碼、有浮水印
\fi

%%%%%%%%%%%%%%%%%%%%%%%%%%%%%%%%%%%%%%%%%%%%%%%%%%%%%%%%%%%%%%%


%%%%%%%%%%%%%%%%%%%%%%%%%%%%%%%%%%%%%%%%%%%%%%%%%%%%%%%%%%%%%%%%%%%%%
%%%%%%%%%%%%%%%%%%%%%%%%%%%%%%%
%       論文口試委員審定書 (計頁碼,但不印頁碼) 
%%%%%%%%%%%%%%%%%%%%%%%%%%%%%%%
%
% insert the printed standard form when the thesis is ready to bind
% 在口試完成後,再將已簽名的審定書放入以便裝訂
% create an entry in table of contents for 審定書
% 目前送出空白頁

%\newpage{\thispagestyle{empty}\addcontentsline{toc}{chapter}{\nameCommitteeForm}\mbox{}\clearpage}%


%%%%%%%%%%%%%%%%%%%%%%%%%%%%%%%
%       中文摘要 
%%%%%%%%%%%%%%%%%%%%%%%%%%%%%%%
%
% aligned to the center of the page
\chapter*{\mdseries\nameCabstract}
% create an entry in table of contents for 中文摘要
% \addcontentsline{toc}{chapter}{ \nameCabstract}
\addcontentsline{toc}{chapter}{\nameCabstract}
\thispagestyle{plain}  % 無 header,但在浮水印模式下會有浮水印
% \vspace*{0.5cm}
% Resume the line spacing to the desired setting
\renewcommand{\baselinestretch}{\mybaselinestretch}   %恢復原設定
%it needs a font size changing command to be effective
% restore the font size to normal
\normalsize
%%%%%%%%%%%%%
\input{frontpages/my_cabstract.tex}

%%%%%%%%%%%%%%%%%%%%%%%%%%%%%%%
%       英文摘要 
%%%%%%%%%%%%%%%%%%%%%%%%%%%%%%%
%
% aligned to the center of the page
\chapter*{\mdseries\nameEabstract}
% create an entry in table of contents for 中文摘要
% \addcontentsline{toc}{chapter}{ \nameEabstract}
\addcontentsline{toc}{chapter}{\nameEabstract}
\thispagestyle{plain}  % 無 header,但在浮水印模式下會有浮水印
% \vspace*{0.5cm}
% Resume the line spacing to the desired setting
\renewcommand{\baselinestretch}{\mybaselinestretch}   %恢復原設定
%it needs a font size changing command to be effective
% restore the font size to normal
\normalsize
%%%%%%%%%%%%%
\input{frontpages/my_eabstract.tex}

%%%%%%%%%%%%%%%%%%%%%%%%%%%%%%%
%       誌謝 
%%%%%%%%%%%%%%%%%%%%%%%%%%%%%%%
%
% Acknowledgment
% aligned to the center of the page
\chapter*{\mdseries\nameAckn}
% create an entry in table of contents for 中文摘要
% \addcontentsline{toc}{chapter}{ \nameAckn}
\addcontentsline{toc}{chapter}{\nameAckn}
\thispagestyle{plain}  % 無 header,但在浮水印模式下會有浮水印
% \vspace*{0.5cm}
% Resume the line spacing to the desired setting
\renewcommand{\baselinestretch}{\mybaselinestretch}   %恢復原設定
%it needs a font size changing command to be effective
% restore the font size to normal
\normalsize
%%%%%%%%%%%%%
\input{frontpages/my_ackn.tex}

%%%%%%%%%%%%%%%%%%%%%%%%%%%%%%%
%       目錄 
%%%%%%%%%%%%%%%%%%%%%%%%%%%%%%%
%
% Table of contents
\newpage
\renewcommand{\contentsname}{\mdseries\nameToc}
\phantomsection % for hyperref to register this
% \addcontentsline{toc}{chapter}{ \nameToc}
\addcontentsline{toc}{chapter}{\nameToc}
\tableofcontents



%%%%%%%%%%%%%%%%%%%%%%%%%%%%%%%
%       圖目錄 
%%%%%%%%%%%%%%%%%%%%%%%%%%%%%%%
%
% List of Figures
\newcommand{\loflabel}{圖}
\newpage
\renewcommand{\numberline}[1]{\loflabel~#1\hspace*{1em}}
\renewcommand{\listfigurename}{\mdseries\nameTof}
\phantomsection % for hyperref to register this
% \addcontentsline{toc}{chapter}{ \nameTof}
\addcontentsline{toc}{chapter}{\nameTof}
\listoffigures


%%%%%%%%%%%%%%%%%%%%%%%%%%%%%%%
%       表目錄 
%%%%%%%%%%%%%%%%%%%%%%%%%%%%%%%
%
% List of Tables
\newcommand{\lotlabel}{表}
\newpage
\renewcommand{\numberline}[1]{\lotlabel~#1\hspace*{1em}}
\renewcommand{\listtablename}{\mdseries\nameLot}
\phantomsection % for hyperref to register this
% \addcontentsline{toc}{chapter}{ \nameLot}
\addcontentsline{toc}{chapter}{\nameLot}
\listoftables

%%%%%%%%%%%%%%%%%%%%%%%%%%%%%%%
%       演算法目錄 
%%%%%%%%%%%%%%%%%%%%%%%%%%%%%%%
%
% List of Figures
%\newpage
%\renewcommand{\listalgorithmname}{\protect\makebox[5cm][s]{\nameToa}}
%\makebox{} is fragile; need protect
%\addcontentsline{toc}{chapter}{\nameToa}
%\listofalgorithms


%%%%%%%%%%%%%%%%%%%%%%%%%%%%%%%
%       符號說明 
%%%%%%%%%%%%%%%%%%%%%%%%%%%%%%%
%
% Symbol list
% define new environment, based on standard description environment
% adapted from p.60~64, <<The LaTeX Companion>>, 1994, ISBN 0-201-54199-8
%\newcommand{\SymEntryLabel}[1]%
% {\makebox[3cm][l]{#1}}
%
%\newenvironment{SymEntry}
%   {\begin{list}{}%
%       {\renewcommand{\makelabel}{\SymEntryLabel}%
%        \setlength{\labelwidth}{3cm}%
%        \setlength{\leftmargin}{\labelwidth}%
%        }%
%   }%
%   {\end{list}}
%%
%\newpage
%\chapter*{\protect\makebox[5cm][s]{\nameSlist}} %\makebox{} is fragile; need protect
%\addcontentsline{toc}{chapter}{\nameSlist}
%\input{frontpages/my_symbols.tex}

% Resume the line spacing to the desired setting
\renewcommand{\baselinestretch}{\mybaselinestretch}   %恢復原設定
%it needs a font size changing command to be effective
% restore the font size to normal
\normalsize

\newpage
\setcounter{page}{1}
% \pagenumbering{arabic}
%% 論文本體頁碼回復為阿拉伯數字計頁,並從頭起算
\pagenumbering{arabic}
%%%%%%%%%%%%%%%%%%%%%%%%%%%%%%%% 

	%---------------------------------------------------------------------------------------------------------
	% main body 論文主體。建議以「章」為檔案分割的依據。
	% 以下為建議的命名分類
	%   introduction.tex   related_work.tex  protocol.tex  evaluation.tex  conclusion.tex
	% 做為這幾個「章」的檔案名稱,並將檔案存放於資料夾 sections/ 下
	% 實際命名方式可以隨你意
	% 在撰寫各章草稿時,可以把其他章節關掉 (行首加百分號)
	%\input{example/example_body.tex}  % 所附的範例

	\input{sections/1_introduction.tex}
	\input{sections/2_relative-work.tex}
	% \renewcommand\thetable{\arabic{chapter}-\arabic{table}}
% \renewcommand\thefigure{\arabic{chapter}-\arabic{figure}} 

\chapter{研究方法}
\label{cha:method} 

\section{test3}
\label{sec:test3}
	% \renewcommand\thetable{\arabic{chapter}-\arabic{table}}
% \renewcommand\thefigure{\arabic{chapter}-\arabic{figure}} 

\chapter{實驗結果與分析}
\label{cha:result} 

\section{test4}
\label{sec:test4}
	% \renewcommand\thetable{\arabic{chapter}-\arabic{table}}
% \renewcommand\thefigure{\arabic{chapter}-\arabic{figure}} 

\chapter{實驗設計}
\label{cha:dis} 

\section{test5}
\label{sec:test5}
	\input{sections/6_conclusion.tex}

	%---------------------------------------------------------------------------------------------------------
	% back pages 後頁
	% 包括參考文獻、附錄、自傳
	% 實際內容由
	%    my_bib.bib, my_appendix.tex, my_vita.tex
	% 決定
	% ntust_backpages.tex 此檔只提供整體架構的定義,不需更動
	% 在撰寫各章草稿時,可以把此部份「關掉」,以節省無謂的編譯時間。
        
	%
% this file is encoded in utf-8
% v1


%%% 參考文獻
%%%%%%%%%%%%%%%%%%%%%%%%%%%%%%%
% 中文文獻 bib add keywords = {chinese}
%%%%%%%%%%%%%%%%%%%%%%%%%%%%%%%
\chapter*{\nameRef}
\addcontentsline{toc}{chapter}{\nameRef}
%
\phantomsection %隱藏標記 (因為\section* 不會自動增加節的編號)
\section*{中文部分}
\addcontentsline{toc}{section}{中文部分}
\printbibliography[heading=none, keyword=chinese]
%%%%%%%%%%%%%%%%%%%%%%%%%%%%%%%

%%%%%%%%%%%%%%%%%%%%%%%%%%%%%%%
% 英文文獻 bib add keywords = {english}
%%%%%%%%%%%%%%%%%%%%%%%%%%%%%%%
\newpage
\phantomsection %隱藏標記 (因為\section* 不會自動增加節的編號)
\section*{西文部分}
\addcontentsline{toc}{section}{西文部分}
\printbibliography[heading=none, keyword=english]

% \printbibliography

%%%%%%%%%%%%%%%%%%%%%%%%%%%%%%%

%%% 附錄
%\input{backpages/my_appendix.tex}


\end{document} 
